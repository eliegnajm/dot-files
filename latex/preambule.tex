\usepackage{hyperref} % for links and mailto
\usepackage{xcolor}
\hypersetup{
    bookmarks=true,         % show bookmarks bar?
    unicode=false,          % non-Latin characters in Acrobat’s bookmarks
    pdftoolbar=true,        % show Acrobat’s toolbar?
    pdfmenubar=true,        % show Acrobat’s menu?
    pdffitwindow=false,     % window fit to page when opened
    pdfstartview={FitH},    % fits the width of the page to the window
    pdftitle={My title},    % title
    pdfauthor={Author},     % author
    pdfsubject={Subject},   % subject of the document
    pdfcreator={Creator},   % creator of the document
    pdfproducer={Producer}, % producer of the document
    pdfkeywords={keyword1} {key2} {key3}, % list of keywords
    pdfnewwindow=true,      % links in new window
    colorlinks=true,       % false: boxed links; true: colored links
    linkcolor=orange,          % color of internal links
    citecolor=green,        % color of links to bibliography
    filecolor=magenta,      % color of file links
    urlcolor=cyan           % color of external links
}
\usepackage[cm]{fullpage}		%to get the full page
\setlength{\parindent}{0in}	%disable indentation
\usepackage[all]{hypcap}
\usepackage{morefloats}
\usepackage{titlesec}
\titlespacing{\section}{0pt}{0pt}{5pt}
\renewcommand \thesection{\Roman{section}}



% Make the font times new roman
\renewcommand{\sfdefault}{phv}
\renewcommand{\rmdefault}{ptm}
\renewcommand{\ttdefault}{pcr}


% Use this to remove figure x from caption
% \usepackage[labelformat=empty]{caption}

% Use this to remove page numbering
% \pagestyle{empty}

\usepackage{graphicx}
\newcommand{\myimage}[4]{
  \begin{figure}[htb]
  \begin{center}
  \setlength\fboxrule{0.5pt}
  \fbox{\includegraphics[scale=#1]{#2}}
  \end{center}
  \vspace{-17pt}
  \caption{#3}
  \label{fig:#4}
  \end{figure}
}
\newcommand{\mysimpleimage}[2]{
  \begin{figure}[htb]
  \begin{center}
  \setlength\fboxrule{0.5pt}
  \fbox{\includegraphics[scale=#1]{#2}}
  \end{center}
  \vspace{-17pt}
  \end{figure}
}
\newcommand{\myframedtext}[3]{
  \begin{figure}[htb]
  \footnotesize
  \begin{framed}
  #1
  \end{framed}
  \normalsize
  \vspace{-17pt}
  \caption{#2}
  \label{fig:#3}
  \end{figure}
}

\usepackage{framed}
%& -output-directory=out 


\newcommand{\HRule}{\rule{0.6\linewidth}{0.4mm}}

% Listing env reformating, uncomment to use!

%\usepackage{listings}
% \lstset{ %
% language=Octave,                % the language of the code
% basicstyle=\footnotesize,       % the size of the fonts that are used for the code
% numbers=left,                   % where to put the line-numbers
% numberstyle=\footnotesize,      % the size of the fonts that are used for the line-numbers
% stepnumber=1,                   % the step between two line-numbers. If it's 1, each line 
% % will be numbered
% numbersep=5pt,                  % how far the line-numbers are from the code
% backgroundcolor=\color{white},  % choose the background color. You must add \usepackage{color}
% showspaces=false,               % show spaces adding particular underscores
% showstringspaces=false,         % underline spaces within strings
% showtabs=false,                 % show tabs within strings adding particular underscores
% frame=single,                   % adds a frame around the code
% tabsize=2,                      % sets default tabsize to 2 spaces
% captionpos=b,                   % sets the caption-position to bottom
% breaklines=true,                % sets automatic line breaking
% breakatwhitespace=false,        % sets if automatic breaks should only happen at whitespace
% title=\lstname,                 % show the filename of files included with \lstinputlisting;
% % also try caption instead of title
% escapeinside={\%*}{*)},         % if you want to add a comment within your code
% morekeywords={*,...}            % if you want to add more keywords to the set
% }
